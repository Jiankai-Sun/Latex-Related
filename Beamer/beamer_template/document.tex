\documentclass[xcolor=svgnames]{beamer}
%[12pt,handout] voor handouts
%[notes=only] voor alleen de notes
%[notes=show] voor handouts plus notes

\mode<presentation>

\usetheme{Malmoe}
%The general lay-out for the presentation. Other options: Berkeley, Warsaw, Boadilla, Goettingen, Hannover, Luebeck, Madrid, Malmoe, Montpellier, Pittsburgh, %Rochester

\setbeamercolor*{titlelike}{parent=structure}

\usecolortheme{dove} 
%The colortheme for the presentation. Other options are: beaver, dolphin, dove, seahorse, whale.

\usecolortheme[named=DarkSlateGrey]{structure} 
%The color of headings and bullets in the presentation. Other options are: DarkSlateGray,SeaGreen, there's more to be found on the internet.

\usetheme{split}

\useoutertheme{tree} 
%The way the outline of the presentation is shown in header. Other options are: infolines, miniframes, shadow, smoothbars, split, tree

\useinnertheme{rectangles}
%The shape of bullets in the presentation. Other options are: rectangles, circles

\setbeamertemplate{blocks}[shadow=false] 
%The way blocks are shown.

\beamertemplatenavigationsymbolsempty 
%Remove the navigationsymbols from the presentation.

\setbeamercovered{transparent}

\usepackage[english]{babel} 
%The language used to hyphenate

\usepackage{graphicx} 
%To include pictures

%\usepackage{hyperref} 
%To include links

\graphicspath{{images/}} 
%To redirect to a folder named 'images' within the document folder where pictures can be stored.

\setbeamercolor{alerted text}{fg=orange} 
%The color of alerted text

\usepackage{amssymb,amsmath}
%To include formulas

\usepackage{lmodern} 
%The font. Other possibilities are: palatino, inconsolata, gfsdidot, comfortaa, gfsneohellenic, lmodern, there's more to be found on the internet

\renewcommand*\ttdefault{cmvtt}
\usepackage[T1]{fontenc}
%\hypersetup{pdfstartview={Fit}}

\AtBeginSection[]
{
\begin{frame}<beamer>{Outline}
\tableofcontents[currentsection]
\end{frame} % creates an outline overview before each section
}

\title[short title]{\textbf{LONG TITLE lalalalalalalala}
\\[0.3em] \large{blablabla}} 
%title and subtitle
%short titles between [] are shown in header and footer

\author[shortname1 \and shortname2] {
	longname1\inst{1,3} \and longname2\inst{2} 
} 
% authors
 
\institute{
    \inst{1} Department of ladida\\institute1
    \and \inst{2} department of bliblibleu\\ institute2
    \and\inst{3} Faculty lalala\\ institute3
} 
% institutes

\date{}

\begin{document}

\frame{\titlepage} 
%creates titlepage

\frame{\tableofcontents} 
%creates table of contents

\section[shorttitlesec1]{Section1: examples of `animations'} 
%creates section, to be found in table of contents, if preferred to leave out of table of contents mark with * (\section*{notincluded})

\begin{frame}{Frame 1: example of uncover, only, and visible}

\uncover<1>{
uncover
}

\visible<1>{
visible
}

\only<1>{
only
}

\uncover<2>{
uncover is grey when it is not `active' \\
only is completely overwritten when it is not `active' \\
visible is no longer visible but still occupies its original space}
\end{frame}

\note{
Can be used to create notes for presenter. Can be printed by editing the options in the class definition at the top of the document.
}

\begin{frame}{Frame 2: example of itemize uncovering}

\begin{itemize}[<+->]
  \item hello
  \item reader
\end{itemize}

\end{frame}


\section[shorttitlesec2]{Section 2: examples of ways to arrange your slide}

\begin{frame}{Frame 3: example of columns}

\begin{columns}
  \column{10em}
bla
  \column{10em}
bla
\end{columns}
\end{frame}

\note{

notes

}

\begin{frame}{Frame 4: example of blocks}
  \begin{block}{title 1}
    ladida
  \end{block}
  \begin{block}{title 2}
  didali
  \end{block}
\end{frame}

\subsection[shorttitlesubsec]{Miscellaneous}

\begin{frame}[shrink]{Shrink frame and insert media}

If you have added a lot of text on the slide, you can shrink the text by adding [shrink] after the frame command.
Hyperlinks to movies and adding pictures can be achieved similar to the way that works in other LaTeX documents, using includegraphics en hyperref url
\end{frame}

\end{document}